\chapter{Formule}
\section{Fattoriale}
\citaoeis{A000142}
\begin{equation}
n!=1\cdot 2\cdot 3\dots (n-1)\cdot n
\end{equation}\index{Numero!Fattoriale}
\section{Numeri di Fibonacci}
\citaoeis{A000045}
\begin{equation}
F_n=F_{n-1}+F_{n-2}\qquad\ F_{1}=1\; F_{2}=1 
\end{equation}\index{Numero!Fibonacci}
\section{Numeri di Lucas}
\citaoeis{A000032}
\begin{equation}
L_n=L_{n-1}+L_{n-2}\qquad\ L_{1}=2\; L_{2}=1 
\end{equation}\index{Numero!Lucas}
\section{Terne pitagoriche primitive}
\subsection{Terna pitagorica}
\begin{definizione}
$a$, $b$ e $c$ sono una terna pitagorica se \[a^2+b^2=c^2\]
\end{definizione}
\begin{definizione}
Una terna pitagorica  $a$, $b$ e $c$ è una terna primitiva se
\[\mcd(a,b,c)=1\]
\end{definizione}
\begin{teorema}
	Se $m$ e $n$ sono coprimi e sono uno pari e l'altro dispari allora
	\begin{align*}
	a=&m^2-n^2\\
	b=&2mn\\
	c=&m^2+n^2
	\end{align*}
	è una terna primitiva e viceversa. 
\end{teorema}
\begin{proof}
	Dimostriamo che è una terna pitagorica cioè che \[a^2+b^2=c^2\]
	ora 
		\begin{align*}
	a=&m^4-2m^2n^2+n^4\\
	b=&4m^2n^2\\
	c=&m^4+2m^2n^2+n^4\\
	m^4-2m^2n^2+n^4+4m^2n^2=&m^4+2m^2n^2+n^4\\
	\end{align*}
	da cui la dimostrazione della prima parte. 
	
	Dimostriamo che sono primi
\end{proof}
\begin{teorema}
	Se $m$ è un numero dispari allora
		\begin{align*}
	a=&m\\
	b=&\dfrac{m^2-1}{2} \\
	c=&\dfrac{m^2+1}{2}
	\end{align*}
	è una terna pitagorica.\par
		Se $m$ è un numero pari non prodotto fra un pari e un dispari, allora
	\begin{align*}
	a=&m\\
	b=&\dfrac{m^2-4}{4} \\
	c=&\dfrac{m^2+4}{4}
	\end{align*}
		è una terna pitagorica.
\end{teorema}